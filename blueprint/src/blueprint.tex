\input{../../.lake/build/dressed/library/OSforGFF.tex}

\usepackage{amsmath, amsthm, amssymb}
\usepackage{hyperref}

% Theorem environments
\theoremstyle{definition}
\newtheorem{definition}{Definition}[section]
\newtheorem{theorem}{Theorem}[section]
\newtheorem{proposition}{Proposition}[section]
\newtheorem{lemma}{Lemma}[section]
\newtheorem{corollary}{Corollary}[section]
\newtheorem{remark}{Remark}[section]
\newtheorem{axiom_env}{Axiom}[section]

% QFT notation macros
\newcommand{\R}{\mathbb{R}}
\newcommand{\C}{\mathbb{C}}
\newcommand{\Z}{\mathbb{Z}}
\newcommand{\N}{\mathbb{N}}
\newcommand{\calS}{\mathcal{S}}

\title{Osterwalder-Schrader Axioms for the Gaussian Free Field}
\author{Michael R. Douglas, Sarah Hoback, Anna Mei, Ron Nissim}

\begin{document}
\maketitle

% =============================================================================
% Chapter 1: General Mathematics
% =============================================================================
\chapter{General Mathematics}\label{ch:general-mathematics}

This chapter develops the mathematical infrastructure needed for the GFF construction. The results here are pure extensions of Mathlib covering functional analysis, Schwartz function theory, special function integrals, and positive-definite kernels. None of these depend on any project-specific definitions.

\section{Functional Analysis}\label{sec:functional-analysis}

This section develops $L^2$ space infrastructure and properties of bilinear forms involving Schwartz functions and singular kernels, including $L^p$ embeddings, bounded multiplier operators, local integrability of singular kernels, and double mollifier convergence.

\inputleanmodule{OSforGFF.FunctionalAnalysis}

\section{Schwartz Functions and Decay Estimates}\label{sec:schwartz-decay}

This section establishes quantitative decay estimates for Schwartz functions and their bilinear pairings with singular kernels, including polynomial decay bounds, convolution decay, and bilinear translation decay.

\inputleanmodule{OSforGFF.QuantitativeDecay}
\inputleanmodule{OSforGFF.SchwartzTranslationDecay}

\section{Special Functions and Integrals}\label{sec:special-functions}

This section provides closed-form evaluations of special integrals arising in quantum field theory, particularly Fourier transforms of Lorentzian and exponential decay functions, and the Laplace-type integral appearing in the Schwinger representation.

\inputleanmodule{OSforGFF.FourierTransforms}
\inputleanmodule{OSforGFF.LaplaceIntegral}
\inputleanmodule{OSforGFF.BesselFunction}

\section{$L^2$ Time Integral Estimates}\label{sec:l2-time-integral}

This section provides measure-theoretic estimates for time averages of stochastic processes, used in the ergodicity arguments (OS axiom 4).

\inputleanmodule{OSforGFF.L2TimeIntegral}

\section{Positive-Definite Kernels and Matrix Theory}\label{sec:positive-definite-kernels}

This section develops the theory of positive-definite functions and matrices needed for the Gaussian measure construction, including the Schur product theorem, Hadamard exponentials, Gaussian RBF positive definiteness, and abstract positive-definite function type classes.

\inputleanmodule{OSforGFF.FrobeniusPositivity}
\inputleanmodule{OSforGFF.SchurProduct}
\inputleanmodule{OSforGFF.HadamardExp}
\inputleanmodule{OSforGFF.PositiveDefinite}
\inputleanmodule{OSforGFF.GaussianRBF}

\section{Nuclear Space}\label{sec:nuclear-space}

A small number of standard mathematical results are assumed as axioms (via \texttt{axiom}) because their proofs would require substantial Mathlib extensions. The nuclear space characterization is defined via Hilbert--Schmidt embedding, and the nuclearity of Schwartz space is assumed.

\inputleanmodule{OSforGFF.NuclearSpace}

% =============================================================================
% Chapter 2: Basic Definitions
% =============================================================================
\chapter{Basic Definitions}\label{ch:basic-definitions}

This chapter introduces the core type definitions and infrastructure for the Euclidean quantum field theory formalization: spacetime geometry, symmetry groups, test function spaces, Schwartz space integration, and generating functionals.

\section{Spacetime and Symmetries}\label{sec:spacetime-symmetries}

The spacetime is $\R^4$ with the standard Euclidean metric, modeled as $\texttt{EuclideanSpace}\ \R\ (\texttt{Fin}\ 4)$. The coordinate index 0 is the time direction. The Euclidean symmetry group $E(4) = O(4) \rtimes \R^4$ acts on spacetime by rotations and translations, with time reflection $\Theta: (t,\mathbf{x}) \mapsto (-t,\mathbf{x})$ as a distinguished involution.

\inputleanmodule{OSforGFF.Basic}
\inputleanmodule{OSforGFF.Euclidean}
\inputleanmodule{OSforGFF.DiscreteSymmetry}
\inputleanmodule{OSforGFF.SpacetimeDecomp}

\section{Test Function Spaces}\label{sec:test-function-spaces}

Test functions are Schwartz-class functions on $\R^4$, serving as the ``smearing functions'' that pair with distributional field configurations. The formalization uses Mathlib's \texttt{SchwartzMap} type, with real, complex, and positive-time variants. Time translation is formalized in detail because of its role in the ergodicity axiom (OS4).

\inputleanmodule{OSforGFF.ComplexTestFunction}
\inputleanmodule{OSforGFF.PositiveTimeTestFunction_real}
\inputleanmodule{OSforGFF.TimeTranslation}

\section{Schwartz Space Integration}\label{sec:schwartz-integration}

Products of Schwartz functions with singular kernels arise throughout the construction. This section provides integrability results for such products, enabling the use of Fubini's theorem and change-of-variables.

The key technical result is the Tonelli-type identity for spacetime integrals: when $K(t_1, t_2)$ is a bounded measurable kernel depending only on time coordinates,
$$\iint \|f(x)\|\, \|g(y)\|\, K(x_0, y_0)\, dx\, dy = \iint K(t_1, t_2)\, G_f(t_1)\, G_g(t_2)\, dt_1\, dt_2$$
where $G_f(t) = \int_{\R^3} \|f(t, v)\|\, dv$ is the spatial marginal.

For positive-time-supported Schwartz functions, the vanishing at $t = 0$ gives the bound $|f(t,x)| \le Ct/(1+\|x\|)^4$ for $t > 0$, which combines the fundamental theorem of calculus with rapid spatial decay.

\inputleanmodule{OSforGFF.SchwartzProdIntegrable}
\inputleanmodule{OSforGFF.SchwartzTonelli}

\section{Generating Functionals}\label{sec:generating-functionals}

The generating functional $Z[J] = \int e^{i\langle\omega,J\rangle}\, d\mu(\omega)$ encodes all correlation functions of the field theory. For a Gaussian measure with covariance $C$, it takes the explicit form $Z[J] = e^{-\frac{1}{2}C(J,J)}$.

The Schwinger $n$-point functions are defined as
$$S_n(f_1, \ldots, f_n) = \int \langle\omega,f_1\rangle \cdots \langle\omega,f_n\rangle\, d\mu(\omega)$$
For the 2-point function, $S_2(f,g) = \int \langle\omega,f\rangle\, \langle\omega,g\rangle\, d\mu$.

The Gaussian property requires $Z[J] = e^{-\frac{1}{2}S_2(J,J)}$, i.e., all correlation functions are determined by the 2-point function via Wick's theorem.

The smeared-to-distributional 2-point function is recovered by expressing the smeared 2-point function as a double convolution of bump functions with the kernel $C$, then applying the double mollifier convergence theorem from Section~\ref{sec:functional-analysis}. As the bump support shrinks to zero, the smeared function converges pointwise to $C(x)$ for $x \ne 0$.

\inputleanmodule{OSforGFF.Schwinger}
\inputleanmodule{OSforGFF.SchwingerTwoPointFunction}

% =============================================================================
% Chapter 3: Free Covariance
% =============================================================================
\chapter{Free Covariance}\label{ch:free-covariance}

The free massive scalar field propagator $C(x, y)$ and its properties form the analytical core of the GFF construction. The construction proceeds from momentum space through Fourier transform to position space, establishing all the analytic properties needed for the OS axioms.

\section{Momentum Space and Schwinger Representation}\label{sec:covariance-momentum}

The free Euclidean propagator for a massive scalar field with mass $m > 0$ in $d = 4$ dimensions is defined in momentum space as
$$\hat{C}(k) = \frac{1}{\|k\|^2 + m^2}$$
The position-space kernel is obtained by inverse Fourier transform:
$$C(x - y) = \frac{1}{(2\pi)^4} \int_{\R^4} \frac{e^{ik\cdot(x-y)}}{\|k\|^2 + m^2}\, d^4k$$
In 4D, this evaluates to the Bessel function representation:
$$C(x, y) = \frac{m}{4\pi^2 \|x - y\|} K_1(m\|x - y\|)$$
where $K_1$ is the modified Bessel function of the second kind.

The propagator admits a proper-time (Schwinger) representation:
$$\frac{1}{\|k\|^2 + m^2} = \int_0^{\infty} e^{-t(\|k\|^2 + m^2)}\, dt$$
Performing the Fourier transform under the proper-time integral yields the heat kernel representation:
$$C(x, y) = \int_0^{\infty} e^{-tm^2} H_t(\|x-y\|)\, dt$$
where $H_t(r) = \frac{1}{16\pi^2 t^2} e^{-r^2/(4t)}$ is the 4D heat kernel.

To rigorously justify Fubini-type exchanges of integration, a Gaussian regulator $e^{-\alpha\|k\|^2}$ is introduced:
$$C_\alpha(x, y) = \int \frac{e^{-\alpha\|k\|^2} e^{ik\cdot(x-y)}}{\|k\|^2 + m^2} \frac{dk}{(2\pi)^4}$$
This makes all integrals absolutely convergent. The true covariance is recovered as $\alpha \to 0^+$.

\inputleanmodule{OSforGFF.CovarianceMomentum}

\section{Parseval Identity and Positive Semidefiniteness}\label{sec:parseval}

For test functions $f, g \in \calS(\R^4, \C)$, the bilinear covariance form is:
$$\langle f, Cg\rangle = \iint \overline{f(x)}\, C(x-y)\, g(y)\, dx\, dy$$
The \emph{Parseval identity} relates this to momentum space:
$$\mathrm{Re}\langle f, Cf\rangle = \int \frac{|\hat{f}(k)|^2}{\|k\|^2 + m^2}\, dk$$
This immediately gives positive semi-definiteness: $\mathrm{Re}\langle f, Cf\rangle \ge 0$ since the integrand is non-negative.

\inputleanmodule{OSforGFF.Parseval}

\section{Covariance Properties and Hilbert Space Embedding}\label{sec:covariance-properties}

The covariance bilinear form has important structural properties: it is symmetric ($\langle f, Cg\rangle = \langle g, Cf\rangle$), additive and homogeneous in each argument, Euclidean-invariant ($\langle g\cdot f, C(g\cdot h)\rangle = \langle f, Ch\rangle$), and invariant under time reflection ($C(\Theta x, \Theta y) = C(x, y)$).

The covariance form factors through a Hilbert space. Define the map $T\colon \texttt{TestFunction} \to L^2$ by
$$Tf(k) = \hat{f}(k) \sqrt{\frac{1}{\|k\|^2 + m^2}}$$
Then $C(f, f) = \|Tf\|^2$, establishing the Hilbert space embedding required by the Minlos theorem.

\inputleanmodule{OSforGFF.Covariance}
\inputleanmodule{OSforGFF.CovarianceR}

% =============================================================================
% Chapter 4: Gaussian Measure Construction
% =============================================================================
\chapter{Gaussian Measure Construction}\label{ch:gaussian-measure}

This chapter constructs the Gaussian Free Field (GFF) measure on the space of distributional field configurations and verifies its Gaussian property. The GFF measure $\mu_{\mathrm{GFF}}$ is the probability measure on the space of tempered distributions $\calS'(\R^4)$ characterized by its characteristic functional:
$$\int e^{i\langle\omega,f\rangle}\, d\mu(\omega) = e^{-\frac{1}{2}C(f,f)}$$
where $C(f, g)$ is the free covariance bilinear form.

\section{Minlos Theorem}\label{sec:minlos}

The construction uses the \emph{Minlos theorem}: a continuous, positive-definite functional on a nuclear space is the characteristic functional of a unique probability measure on the dual space.

\inputleanmodule{OSforGFF.Minlos}

\section{Analytic Properties}\label{sec:analytic-properties}

The negation map $\omega \mapsto -\omega$ is measurable, and the GFF measure is invariant under it. This yields the vanishing of the first moment: $\int\langle\omega,a\rangle\, d\mu = 0$ for all test functions $a$.

\inputleanmodule{OSforGFF.MinlosAnalytic}

\section{GFF Measure Construction}\label{sec:gff-construction}

Apply the Minlos theorem to the free covariance to obtain a probability measure $\mu$ on $\texttt{FieldConfiguration}$ satisfying the characteristic function identity.

\inputleanmodule{OSforGFF.GFFMconstruct}

\section{Gaussian Moments}\label{sec:gaussian-moments}

The integrability of $\langle\omega,\phi\rangle\langle\omega,\psi\rangle$ for complex test functions follows from Fernique's theorem, $L^p$ membership via H\"older's inequality, and the Cauchy--Schwarz inequality.

\inputleanmodule{OSforGFF.GaussianMoments}

\section{Gaussianity Verification}\label{sec:gaussianity}

The extension from real to complex characteristic functions is the most technically involved step. For fixed real test functions $f, g$, both sides of the identity $Z(z_0, z_1) = e^{-\frac{1}{2}C(z_0 f + z_1 g,\, z_0 f + z_1 g)}$ are entire functions of $(z_0, z_1) \in \C^2$.
\begin{itemize}
\item \emph{Left side:} The integral $\int \exp(i\langle\omega, z_0 f + z_1 g\rangle)\, d\mu$ is analytic by Fernique's theorem (sub-Gaussian integrability).
\item \emph{Right side:} Analytic because it is the composition of the exponential with a polynomial.
\end{itemize}
They agree on $\R^2$ by the real characteristic functional identity. By analytic continuation (the identity theorem, applied slice by slice), they agree on $\C^2$. Any complex test function $J$ can be decomposed as $z_0 f + z_1 g$ where $f = \mathrm{Re}(J)$, $g = \mathrm{Im}(J)$, $z_0 = 1$, $z_1 = i$.

From the complex Gaussian identity $Z[J] = e^{-\frac{1}{2}S_2(J,J)}$, the 2-point function is extracted by differentiating twice: $S_2(f, g) = C(f, g)$.

\inputleanmodule{OSforGFF.GFFIsGaussian}
\inputleanmodule{OSforGFF.GaussianFreeField}

% =============================================================================
% Chapter 5: OS Axiom Definitions
% =============================================================================
\chapter{OS Axiom Definitions}\label{ch:os-axiom-definitions}

The Osterwalder--Schrader (OS) axioms characterize Euclidean quantum field theories that admit analytic continuation to relativistic (Minkowski) QFTs via the Wick rotation. They were introduced by Osterwalder and Schrader in 1973--1975, building on earlier work of Nelson and Symanzik.

The axioms are stated in terms of the \emph{generating functional} (or characteristic functional):
$$Z[f] = \int_{\Omega} e^{i\langle \omega, f \rangle}\, d\mu(\omega)$$
where $\Omega = \calS'(\R^4)$ is the space of tempered distributions (field configurations), $\mu$ is a probability measure on $\Omega$, $f$ is a test function in $\calS(\R^4)$, and $\langle \omega, f \rangle$ denotes the distributional pairing.

This project follows the Glimm--Jaffe formulation and proves the axioms for the Gaussian Free Field with mass $m > 0$ in 4-dimensional Euclidean spacetime.

\section{OS0: Analyticity}\label{sec:os0-def}

For any finite collection of complex test functions $J_1, \ldots, J_n \in \calS(\R^4, \C)$, the map
$$z = (z_1, \ldots, z_n) \mapsto Z\left[\sum_{i=1}^n z_i J_i\right]$$
is analytic on $\C^n$. Analyticity ensures that the Schwinger functions are well-defined as functional derivatives of $Z$, and that the theory can be analytically continued from Euclidean to Minkowski signature via the Wick rotation.

\section{OS1: Regularity}\label{sec:os1-def}

There exist $p \in [1, 2]$ and $c > 0$ such that for all $f \in \calS(\R^4, \C)$:
$$|Z[f]| \le \exp\left(c \cdot \left(\|f\|_{L^1} + \|f\|_{L^p}^p\right)\right)$$
When $p = 2$, the two-point Schwinger function $S_2(x)$ must be locally integrable. This bound ensures the field theory measure is well-defined and the Schwinger functions are tempered distributions.

\section{OS2: Euclidean Invariance}\label{sec:os2-def}

For all $g \in E(4)$ and $f \in \calS(\R^4, \C)$:
$$Z[g \cdot f] = Z[f]$$
where $(g \cdot f)(x) = f(g^{-1} x)$ is the pullback action. The vacuum state respects Euclidean symmetry. After Wick rotation, this becomes Poincar\'e invariance of the Minkowski theory.

\section{OS3: Reflection Positivity}\label{sec:os3-def}

Let $\Theta$ denote time reflection: $\Theta(t, \vec{x}) = (-t, \vec{x})$. For test functions $f_1, \ldots, f_n$ supported in the positive time half-space $\{t > 0\}$ and real coefficients $c_1, \ldots, c_n$:
$$\sum_{i,j} c_i\, c_j \cdot \mathrm{Re}\left(Z[f_i - \Theta f_j]\right) \ge 0$$
This is the key axiom for the OS reconstruction theorem. The positive semi-definite form becomes an inner product after quotienting by null vectors, yielding the physical Hilbert space of the QFT.

\section{OS4: Clustering and Ergodicity}\label{sec:os4-def}

The clustering axiom states that for test functions $f, g$ and any $\varepsilon > 0$, there exists $R > 0$ such that for spatial separation $\|a\| > R$:
$$|Z[f + T_a g] - Z[f] \cdot Z[g]| < \varepsilon$$
Distant regions of spacetime become statistically independent, which is equivalent to uniqueness of the vacuum state.

The ergodicity formulation states that for generating-function observables $A(\varphi) = \sum_j z_j e^{\langle\varphi, f_j\rangle}$, the time average converges to the expectation in $L^2(\mu)$:
$$\lim_{T \to \infty} \int_\Omega \left|\frac{1}{T}\int_0^T A(T_s \varphi)\, ds - \mathbb{E}_\mu[A]\right|^2 d\mu(\omega) = 0$$

A quantitative polynomial clustering variant is also established: for any $\alpha > 0$, there exists $c \ge 0$ such that for all $s \ge 0$:
$$\left|\mathbb{E}_\mu\left[e^{\langle\varphi,f\rangle + \langle T_s\varphi, g\rangle}\right] - \mathbb{E}_\mu\left[e^{\langle\varphi,f\rangle}\right]\mathbb{E}_\mu\left[e^{\langle\varphi,g\rangle}\right]\right| \le c\, (1 + s)^{-\alpha}$$

\inputleanmodule{OSforGFF.OS_Axioms}

% =============================================================================
% Chapter 6: OS0 — Analyticity
% =============================================================================
\chapter{OS0 --- Analyticity}\label{ch:os0}

OS0 (Analyticity) is the first Osterwalder--Schrader axiom. It states that the generating functional $Z[f]$ depends analytically on the test function arguments. This is essential for the Wick rotation connecting Euclidean and Minkowski signatures, for defining correlation functions as functional derivatives of $Z$, and for perturbation theory via Taylor expansions.

\section{Proof Strategy}\label{sec:os0-strategy}

The proof applies a \emph{holomorphic integral theorem}: if the integrand $f(z, \omega) = \exp(i\sum_i z_i \langle\omega, J_i\rangle)$ satisfies appropriate measurability, analyticity, integrability, and derivative bound conditions in $z$ and $\omega$, then the integral $F(z) = \int f(z, \omega)\, d\mu(\omega)$ is holomorphic. Goursat's theorem in $n$ dimensions then converts holomorphy to analyticity. The proof requires one project-specific axiom: Goursat's theorem in $n$ dimensions, which is standard in several complex variables but not yet available in Mathlib.

\inputleanmodule{OSforGFF.OS0_GFF}

% =============================================================================
% Chapter 7: OS1 — Regularity
% =============================================================================
\chapter{OS1 --- Regularity}\label{ch:os1}

OS1 (Regularity) establishes that the generating functional $Z[f]$ has controlled growth as a function of the test function $f$. For the GFF with mass $m > 0$, the proof establishes the bound with $p = 2$ and $c = 1/(2m^2)$.

\section{Exponential Bound}\label{sec:os1-exponential}

The GFF generating functional has the closed form $Z[f] = \exp(-\tfrac{1}{2}\langle f, Cf\rangle_{\C})$. The proof decomposes the sesquilinear form, applies the Parseval identity in momentum space, and uses the Plancherel theorem to obtain the $L^2$ bound. The two-point local integrability follows from Bessel function asymptotics. The OS1 proof uses zero project-specific axioms.

\inputleanmodule{OSforGFF.OS1_GFF}

% =============================================================================
% Chapter 8: OS2 — Euclidean Invariance
% =============================================================================
\chapter{OS2 --- Euclidean Invariance}\label{ch:os2}

OS2 (Euclidean Invariance) requires that the quantum field theory has no preferred direction or origin in Euclidean spacetime. After Wick rotation, Euclidean invariance becomes Poincar\'e invariance of the Minkowski theory.

\section{Proof Strategy}\label{sec:os2-strategy}

The proof factors into a general Gaussian result (any Gaussian measure with Euclidean-invariant covariance satisfies OS2) and the GFF-specific verification that the free covariance depends only on $\|x - y\|$. This is the cleanest OS axiom proof, requiring zero project-specific axioms.

\inputleanmodule{OSforGFF.OS2_GFF}

% =============================================================================
% Chapter 9: OS3 — Reflection Positivity
% =============================================================================
\chapter{OS3 --- Reflection Positivity}\label{ch:os3}

Reflection positivity (OS3) is the most technically demanding of the Osterwalder--Schrader axioms. It is the Euclidean manifestation of \emph{unitarity} --- the requirement that quantum mechanics has positive probabilities and the Hilbert space has a positive-definite inner product. Without OS3, a Euclidean field theory cannot be analytically continued to define a physical relativistic QFT.

The proof has two main parts:
\begin{enumerate}
\item \textbf{Covariance reflection positivity via Fourier analysis:} Show $\mathrm{Re}\langle \Theta f, Cf \rangle \ge 0$ for each positive-time $f$ by rewriting the bilinear form in momentum space and factorizing into a squared norm.
\item \textbf{Matrix assembly via the Hadamard (Schur) product theorem:} Extend from a single test function to the general $n$-function case using Gaussian factorization and the fact that entrywise exponentials of PSD matrices remain PSD.
\end{enumerate}

\section{Covariance Reflection Positivity}\label{sec:os3-covariance-rp}

The goal is to show $\mathrm{Re}\langle\Theta f, f\rangle_C \ge 0$ for any complex test function $f$ with $f(x) = 0$ when $x_0 \le 0$. The proof proceeds via the Schwinger representation, Fourier decomposition, evaluation of the proper-time integral to obtain a mixed representation, and factorization using positive-time support to express the result as an integral of $|F_\omega|^2/\omega$, which is pointwise non-negative.

\inputleanmodule{OSforGFF.OS3_MixedRepInfra}
\inputleanmodule{OSforGFF.OS3_MixedRep}
\inputleanmodule{OSforGFF.OS3_CovarianceRP}

\section{Matrix Assembly}\label{sec:os3-matrix-assembly}

The extension from a single test function to the general $n$-function case uses bilinearity to show the reflection covariance matrix is PSD, the Hadamard exponential to preserve PSD under entrywise exponentiation, and Gaussian entry factorization to assemble the final quadratic form inequality.

\inputleanmodule{OSforGFF.OS3_GFF}

% =============================================================================
% Chapter 10: OS4 — Clustering and Ergodicity
% =============================================================================
\chapter{OS4 --- Clustering and Ergodicity}\label{ch:os4}

The OS4 axiom comes in two related forms: \emph{clustering} (distant regions become statistically independent) and \emph{ergodicity} (time averages converge to ensemble averages). For the Gaussian Free Field, clustering follows from the decay of the free propagator, and ergodicity follows from clustering via variance bounds on time averages.

\section{MGF Infrastructure}\label{sec:os4-mgf}

The moment generating functional $M[f] = \mathbb{E}[e^{\langle\phi, f\rangle}]$ is related to the characteristic functional by analytic continuation. For the GFF, $M[f] = e^{\frac{1}{2}C(f,f)}$.

\inputleanmodule{OSforGFF.OS4_MGF}

\section{Clustering}\label{sec:os4-clustering}

The clustering proof exploits the Gaussian factorization formula to reduce the problem to showing the cross covariance $C(f, T_a g) \to 0$ as $\|a\| \to \infty$, which follows from propagator decay. The polynomial clustering variant uses the bilinear translation decay theorem from Chapter~\ref{ch:general-mathematics}.

\inputleanmodule{OSforGFF.OS4_Clustering}

\section{Ergodicity}\label{sec:os4-ergodicity}

The ergodicity proof proceeds through intermediate formulations:
$$\text{OS4'' (Polynomial Clustering, } \alpha=6\text{)} \implies \text{OS4' (Single-function Ergodicity)} \implies \text{OS4 (Full Ergodicity)}$$

The implication chain uses variance bounds on time averages, polynomial decay of the covariance from clustering, and double integral estimates from Chapter~\ref{ch:general-mathematics} to show the variance is $O(1/T) \to 0$.

\inputleanmodule{OSforGFF.OS4_Ergodicity}

% =============================================================================
% Chapter 11: Master Theorem
% =============================================================================
\chapter{Master Theorem}\label{ch:master-theorem}

The master theorem assembles the individual axiom proofs into a single statement: the Gaussian Free Field with mass $m > 0$ in 4-dimensional Euclidean spacetime satisfies all Osterwalder--Schrader axioms.

For any mass parameter $m > 0$, the GFF probability measure $\mu_{\mathrm{GFF}}(m)$ satisfies:
\begin{itemize}
\item \textbf{OS0} (Analyticity): The generating functional is analytic in the test functions (Chapter~\ref{ch:os0}).
\item \textbf{OS1} (Regularity): The generating functional satisfies exponential bounds (Chapter~\ref{ch:os1}).
\item \textbf{OS2} (Euclidean Invariance): The measure is invariant under 4D Euclidean transformations (Chapter~\ref{ch:os2}).
\item \textbf{OS3} (Reflection Positivity): The quadratic form $\sum_{ij} c_i c_j\, \mathrm{Re}(Z[f_i - \Theta f_j]) \ge 0$ (Chapter~\ref{ch:os3}).
\item \textbf{OS4} (Clustering): Distant test functions become statistically independent (Chapter~\ref{ch:os4}).
\item \textbf{OS4} (Ergodicity): Time averages of generating-function observables converge to ensemble averages (Chapter~\ref{ch:os4}).
\end{itemize}

The master theorem in \texttt{GFFmaster.lean} is a short assembly file that imports and combines six independently proved results. Each axiom feeds in via a dedicated proof file, as detailed in the preceding chapters.

The complete proof depends on three axioms declared with the \texttt{axiom} keyword in Lean:
\begin{enumerate}
\item \textbf{Minlos theorem:} A continuous positive-definite normalized functional on a nuclear space is the characteristic functional of a unique probability measure.
\item \textbf{Goursat's theorem in $n$ dimensions:} A holomorphic function $f : \C^n \supset U \to \C$ is analytic (representable by a convergent power series).
\item \textbf{Nuclearity of Schwartz space:} $\calS(\R^n, F)$ is a nuclear topological vector space.
\end{enumerate}
All three are well-established theorems in functional analysis whose full Lean formalization was deferred.

\inputleanmodule{OSforGFF.GFFmaster}

% =============================================================================
% Appendix: Dimension Dependence
% =============================================================================
\appendix
\chapter{Dimension Dependence}\label{ch:dimension-dependence}

This appendix inventories where the spacetime dimension $d = 4$ (and spatial dimension $d-1 = 3$) enters the proofs. The project defines \texttt{STDimension := 4} in \texttt{Basic.lean}; changing this value would require updates to every item listed below as \emph{essential} or \emph{spatial}.

\section{Essential ($d=4$) Dependencies}\label{sec:dim-essential}

These formulas evaluate differently in other dimensions.

\begin{description}
\item[Heat kernel normalization:] The heat kernel uses $(4\pi t)^{-d/2}$, which evaluates to $1/(16\pi^2 t^2)$ for $d=4$.
\item[Bessel representation:] The closed-form covariance $C(x,y) = \frac{m}{4\pi^2 r} K_1(mr)$ is specific to $d=4$. In general dimension $d$, the covariance involves $K_{d/2-1}$.
\item[Schwinger evaluation:] The connection between the proper-time integral and the Bessel form uses the $d=4$ heat kernel.
\item[Plancherel scaling:] The factor $(2\pi)^d$ in Fourier convention changes.
\item[Mixed representation normalization:] The identity $(2\pi)^4 / (2\pi) = (2\pi)^3$ for factoring the mixed representation.
\end{description}

\section{Spatial ($d-1=3$) Dependencies}\label{sec:dim-spatial}

These use the spatial dimension $d-1 = 3$ for integrability.

\begin{description}
\item[Polynomial decay integrability:] $(1+\|x\|)^{-4}$ is integrable in $\R^3$ because the decay rate $4 > d-1 = 3$. In general dimension, the required decay rate would be $> d-1$.
\item[Spatial marginal bounds:] The Schwartz product integrability section uses spatial marginals in $\R^3$.
\item[Local integrability:] The condition $\alpha = 2 < d = 4$ in the two-point local integrability. In general dimension $d$, the singularity is $\|x\|^{-(d-2)}$, so the condition becomes $d-2 < d$, which always holds.
\item[Norm decomposition:] The expansion $\|k\|^2 = k_0^2 + k_1^2 + k_2^2 + k_3^2$ is coupled to $d=4$.
\end{description}

\section{Structural (Dimension-Agnostic) Components}\label{sec:dim-structural}

The following components reference \texttt{STDimension} or \texttt{SpaceTime} but work for any $d \ge 2$: the OS axiom definitions, the Euclidean group structure, time reflection, time translation, spacetime decomposition, complex test function operations, positive-time support, Schwinger functions, the Minlos theorem, the Gaussian measure construction, moment computations, all five OS axiom proofs (OS0--OS4), $L^2$ time integral estimates, the master theorem assembly, 1D Fourier transform identities, Laplace integrals, Hadamard exponentials, Schur product theorem, positive-definite function theory, Gaussian RBF, polynomial decay estimates, bilinear translation decay, and Tonelli for spacetime integrals.

\section{Generalization Notes}\label{sec:dim-generalization}

To port the project to a different dimension $d$:
\begin{enumerate}
\item Change \texttt{STDimension} in \texttt{Basic.lean}.
\item Replace the Bessel $K_1$ representation with the general $K_{d/2-1}$ form.
\item Update the heat kernel normalization $(4\pi t)^{-d/2}$.
\item Update Plancherel scaling factors $(2\pi)^d$.
\item Verify spatial integrability: decay rate must exceed $d-1$.
\item Replace the norm decomposition with the appropriate variant.
\end{enumerate}

\end{document}
