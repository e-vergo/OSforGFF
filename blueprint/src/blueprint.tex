\input{../../.lake/build/dressed/library/OSforGFF.tex}

\usepackage{amsmath, amsthm, amssymb}
\usepackage{hyperref}

% Theorem environments
\theoremstyle{definition}
\newtheorem{definition}{Definition}[section]
\newtheorem{theorem}{Theorem}[section]
\newtheorem{proposition}{Proposition}[section]
\newtheorem{lemma}{Lemma}[section]
\newtheorem{corollary}{Corollary}[section]
\newtheorem{remark}{Remark}[section]
\newtheorem{axiom_env}{Axiom}[section]

% QFT notation macros
\newcommand{\R}{\mathbb{R}}
\newcommand{\C}{\mathbb{C}}
\newcommand{\Z}{\mathbb{Z}}
\newcommand{\N}{\mathbb{N}}
\newcommand{\calS}{\mathcal{S}}

\title{Osterwalder-Schrader Axioms for the Gaussian Free Field}
\author{Michael R. Douglas, Sarah Hoback, Anna Mei, Ron Nissim}

\begin{document}
\maketitle

% =============================================================================
% Chapter 1: General Mathematics
% =============================================================================
\chapter{General Mathematics}

% Background mathematical infrastructure needed for the GFF construction.

\section{Functional Analysis}
\inputleanmodule{OSforGFF.FunctionalAnalysis}

\section{Schwartz Functions and Decay Estimates}
\inputleanmodule{OSforGFF.QuantitativeDecay}
\inputleanmodule{OSforGFF.SchwartzTranslationDecay}

\section{Special Functions and Integrals}
\inputleanmodule{OSforGFF.FourierTransforms}
\inputleanmodule{OSforGFF.LaplaceIntegral}
\inputleanmodule{OSforGFF.BesselFunction}

\section{L2 Time Integral Estimates}
\inputleanmodule{OSforGFF.L2TimeIntegral}

\section{Positive-Definite Kernels and Matrix Theory}
\inputleanmodule{OSforGFF.FrobeniusPositivity}
\inputleanmodule{OSforGFF.SchurProduct}
\inputleanmodule{OSforGFF.HadamardExp}
\inputleanmodule{OSforGFF.PositiveDefinite}
\inputleanmodule{OSforGFF.GaussianRBF}

\section{Nuclear Space}
\inputleanmodule{OSforGFF.NuclearSpace}

% =============================================================================
% Chapter 2: Basic Definitions
% =============================================================================
\chapter{Basic Definitions}

% Core definitions for the Euclidean QFT framework.

\section{Spacetime and Symmetries}
\inputleanmodule{OSforGFF.Basic}
\inputleanmodule{OSforGFF.Euclidean}
\inputleanmodule{OSforGFF.DiscreteSymmetry}
\inputleanmodule{OSforGFF.SpacetimeDecomp}

\section{Test Function Spaces}
\inputleanmodule{OSforGFF.ComplexTestFunction}
\inputleanmodule{OSforGFF.PositiveTimeTestFunction_real}
\inputleanmodule{OSforGFF.TimeTranslation}

\section{Schwartz Space Integration}
\inputleanmodule{OSforGFF.SchwartzProdIntegrable}
\inputleanmodule{OSforGFF.SchwartzTonelli}

\section{Generating Functionals}
\inputleanmodule{OSforGFF.Schwinger}
\inputleanmodule{OSforGFF.SchwingerTwoPointFunction}

% =============================================================================
% Chapter 3: Free Covariance
% =============================================================================
\chapter{Free Covariance}

% Construction and properties of the free covariance kernel.

\inputleanmodule{OSforGFF.CovarianceMomentum}
\inputleanmodule{OSforGFF.Parseval}
\inputleanmodule{OSforGFF.Covariance}
\inputleanmodule{OSforGFF.CovarianceR}

% =============================================================================
% Chapter 4: Gaussian Measure Construction
% =============================================================================
\chapter{Gaussian Measure Construction}

% Construction of the Gaussian measure via Minlos' theorem.

\inputleanmodule{OSforGFF.Minlos}
\inputleanmodule{OSforGFF.MinlosAnalytic}
\inputleanmodule{OSforGFF.GFFMconstruct}
\inputleanmodule{OSforGFF.GaussianMoments}
\inputleanmodule{OSforGFF.GFFIsGaussian}
\inputleanmodule{OSforGFF.GaussianFreeField}

% =============================================================================
% Chapter 5: OS Axiom Definitions
% =============================================================================
\chapter{OS Axiom Definitions}

% Formal statement of the Osterwalder-Schrader axioms.

\inputleanmodule{OSforGFF.OS_Axioms}

% =============================================================================
% Chapter 6: OS0 — Analyticity
% =============================================================================
\chapter{OS0 --- Analyticity}

% Verification of OS0 (analyticity) for the GFF.

\inputleanmodule{OSforGFF.OS0_GFF}

% =============================================================================
% Chapter 7: OS1 — Regularity
% =============================================================================
\chapter{OS1 --- Regularity}

% Verification of OS1 (regularity) for the GFF.

\inputleanmodule{OSforGFF.OS1_GFF}

% =============================================================================
% Chapter 8: OS2 — Euclidean Invariance
% =============================================================================
\chapter{OS2 --- Euclidean Invariance}

% Verification of OS2 (Euclidean invariance) for the GFF.

\inputleanmodule{OSforGFF.OS2_GFF}

% =============================================================================
% Chapter 9: OS3 — Reflection Positivity
% =============================================================================
\chapter{OS3 --- Reflection Positivity}

% Verification of OS3 (reflection positivity) for the GFF.

\inputleanmodule{OSforGFF.OS3_MixedRepInfra}
\inputleanmodule{OSforGFF.OS3_MixedRep}
\inputleanmodule{OSforGFF.OS3_CovarianceRP}
\inputleanmodule{OSforGFF.OS3_GFF}

% =============================================================================
% Chapter 10: OS4 — Clustering and Ergodicity
% =============================================================================
\chapter{OS4 --- Clustering and Ergodicity}

% Verification of OS4 (clustering/ergodicity) for the GFF.

\inputleanmodule{OSforGFF.OS4_MGF}
\inputleanmodule{OSforGFF.OS4_Clustering}
\inputleanmodule{OSforGFF.OS4_Ergodicity}

% =============================================================================
% Chapter 11: Master Theorem
% =============================================================================
\chapter{Master Theorem}

% The GFF satisfies all Osterwalder-Schrader axioms.

\inputleanmodule{OSforGFF.GFFmaster}

\end{document}
