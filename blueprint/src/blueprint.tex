\input{../../.lake/build/dressed/library/OSforGFF.tex}

\usepackage{amsmath, amsthm, amssymb}
\usepackage{hyperref}

% Theorem environments
\theoremstyle{definition}
\newtheorem{definition}{Definition}[section]
\newtheorem{theorem}{Theorem}[section]
\newtheorem{proposition}{Proposition}[section]
\newtheorem{lemma}{Lemma}[section]
\newtheorem{corollary}{Corollary}[section]
\newtheorem{remark}{Remark}[section]
\newtheorem{axiom_env}{Axiom}[section]

% QFT notation macros
\newcommand{\R}{\mathbb{R}}
\newcommand{\C}{\mathbb{C}}
\newcommand{\Z}{\mathbb{Z}}
\newcommand{\N}{\mathbb{N}}
\newcommand{\calS}{\mathcal{S}}

\title{Osterwalder-Schrader Axioms for the Gaussian Free Field}
\author{Michael R. Douglas, Sarah Hoback, Anna Mei, Ron Nissim}

\begin{document}
\maketitle

% =============================================================================
% Chapter 1: General Mathematics
% =============================================================================
\chapter{General Mathematics}\label{ch:general-mathematics}

This chapter develops the mathematical infrastructure needed for the GFF construction. The results here are pure extensions of Mathlib covering functional analysis, Schwartz function theory, special function integrals, and positive-definite kernels. None of these depend on any project-specific definitions.

\section{Functional Analysis}\label{sec:functional-analysis}

This section develops $L^2$ space infrastructure and properties of bilinear forms involving Schwartz functions and singular kernels. The key results include:

\begin{itemize}
\item \textbf{$L^p$ embeddings:} Schwartz functions embed continuously into $L^2$ spaces. The embedding sends $\calS(\R^d, \C)$ into $L^2(\R^d, \C)$.
\item \textbf{Bounded multiplier operators:} If $|g(x)| \le C$ a.e., then $f \mapsto gf$ is a continuous linear map on $L^2$ with $\|gf\|_2 \le C\|f\|_2$.
\item \textbf{Local integrability:} Functions with $|f(x)| \le C\|x\|^{-\alpha}$ and $\alpha < d$ are locally integrable in dimension $d \ge 3$.
\item \textbf{Double mollifier convergence:} For a kernel $C$ continuous away from the origin, the double convolution with approximate identities converges to $C(a)$ as the support shrinks to zero.
\end{itemize}

The double mollifier convergence theorem is a key technical ingredient for connecting the abstract Schwinger 2-point functional to the concrete position-space kernel. Given $C$ continuous on $\{x \ne 0\}$ and $a \ne 0$, with a sequence of bump functions $\varphi_i$ whose support shrinks to zero, the proof proceeds as follows:
\begin{enumerate}
\item Rewrite the double convolution as a single integral against the self-convolution of the bump function.
\item Show the self-convolution is a non-negative approximate identity with integral 1 and support in $B(0, 2r_{\mathrm{out}})$.
\item Apply the general approximate identity convergence theorem for continuous functions.
\end{enumerate}

\inputleanmodule{OSforGFF.FunctionalAnalysis}

\section{Schwartz Functions and Decay Estimates}\label{sec:schwartz-decay}

This section establishes quantitative decay estimates for Schwartz functions and their bilinear pairings with singular kernels. The main results concern:

\begin{itemize}
\item \textbf{Polynomial decay bounds:} Every $f \in \calS$ satisfies $\|f(x)\| \le C(1+\|x\|)^{-N}$ for any $N > 0$.
\item \textbf{Convolution decay:} Convolutions of Schwartz functions with locally integrable kernels decay at any polynomial rate.
\item \textbf{Bilinear translation decay:} $\iint f(x)\, K(x-y)\, g(y-a)\, dx\, dy \to 0$ as $\|a\| \to \infty$, with polynomial rate when $K$ has exponential tail decay.
\end{itemize}

The bilinear decay proofs decompose the kernel $K$ into a singular part (supported on a ball) and a tail part:
$$K(z) = K_{\mathrm{sing}}(z) + K_{\mathrm{tail}}(z)$$
where $K_{\mathrm{sing}}$ is compactly supported and $K_{\mathrm{tail}}$ is bounded. Each piece is handled separately:
\begin{itemize}
\item \emph{Singular part:} Compact support means the convolution with a Schwartz function inherits any polynomial decay rate.
\item \emph{Tail part:} Exponential decay of $K_{\mathrm{tail}}$ combined with polynomial decay of $f$ gives polynomial decay of the convolution.
\end{itemize}
The final result combines both pieces using the triangle inequality.

\inputleanmodule{OSforGFF.QuantitativeDecay}
\inputleanmodule{OSforGFF.SchwartzTranslationDecay}

\section{Special Functions and Integrals}\label{sec:special-functions}

This section provides closed-form evaluations of special integrals arising in quantum field theory, particularly Fourier transforms of Lorentzian and exponential decay functions, and the Laplace-type integral appearing in the Schwinger representation. The key identities proved are:

\begin{itemize}
\item \textbf{Fourier transform of exponential decay:}
$$\int_{-\infty}^{\infty} e^{ikx}\, e^{-\mu|x|}\, dx = \frac{2\mu}{k^2+\mu^2}$$

\item \textbf{Fourier inversion for Lorentzian:}
$$\int_{-\infty}^{\infty} \frac{e^{ikx}}{k^2+\mu^2}\, dk = \frac{\pi}{\mu}\, e^{-\mu|x|}$$

\item \textbf{Laplace integral (Glasser's identity):}
$$\int_0^{\infty} s^{-1/2}\, e^{-a/s - bs}\, ds = \sqrt{\pi/b}\, e^{-2\sqrt{ab}}$$
\end{itemize}

\paragraph{Fourier transforms.} The integrals are evaluated by splitting at 0. On each half-line, $e^{ikx \mp \mu x}$ has a decaying exponential factor whose antiderivative is computed explicitly. The results combine via $|x| = \pm x$ on each half-line.

\paragraph{Laplace integral.} The proof chain is:
\begin{enumerate}
\item Substitution $s = t^2$ reduces to an integral over $(0, \infty)$.
\item Complete the square: $a/t^2 + bt^2 = (\sqrt{a}/t - \sqrt{b}\, t)^2 + 2\sqrt{ab}$.
\item Factor out $e^{-2\sqrt{ab}}$.
\item Glasser's master theorem (a change-of-variables identity) evaluates the remaining Gaussian-like integral as $\sqrt{\pi}/(2\sqrt{b})$.
\end{enumerate}

\inputleanmodule{OSforGFF.FourierTransforms}
\inputleanmodule{OSforGFF.LaplaceIntegral}
\inputleanmodule{OSforGFF.BesselFunction}

\section{$L^2$ Time Integral Estimates}\label{sec:l2-time-integral}

This section provides measure-theoretic estimates for time averages of stochastic processes, used in the ergodicity arguments (OS axiom 4). The main results are:

\begin{itemize}
\item \textbf{Cauchy--Schwarz for set integrals:}
$$\left\|\int_{[a,b]} f\right\|^2 \le (b-a)\int_{[a,b]}\|f\|^2$$

\item \textbf{$L^2$ time average bound:} If $\int \|A(s)\|^2\, d\mu \le M$ for all $s \in [0,T]$, then $\int \|T^{-1}\int_0^T A(s)\, ds\|^2\, d\mu \le M$.

\item \textbf{Product $L^2$ membership:} Joint measurability combined with uniform slice $L^2$ bounds implies $L^2$ membership on the product.

\item \textbf{Double integral polynomial decay:}
$$\int_0^T\int_0^T (1+|s-u|)^{-\alpha}\, ds\, du \le CT \quad\text{for } \alpha > 1.$$
\end{itemize}

\inputleanmodule{OSforGFF.L2TimeIntegral}

\section{Positive-Definite Kernels and Matrix Theory}\label{sec:positive-definite-kernels}

This section develops the theory of positive-definite functions and matrices needed for the Gaussian measure construction. The main results are:

\begin{itemize}
\item \textbf{Schur product theorem:} $A \circ B \succeq 0$ when $A, B \succeq 0$ (Hadamard product of positive semidefinite matrices).
\item \textbf{Hadamard exponential:} $\exp_\circ(R) \succeq 0$ when $R \succeq 0$, proved via the series expansion $\exp_\circ = \sum R^{\circ k}/k!$ and the Schur product theorem.
\item \textbf{Gaussian RBF positive definiteness:} $h \mapsto e^{-\|h\|^2/2}$ is positive definite on any inner product space.
\item \textbf{Abstract positive definiteness:} The type class for translation-invariant positive-definite functions on groups.
\end{itemize}

The Schur product theorem is proved by showing that the Hadamard product of two PSD matrices can be expressed as a principal submatrix of the Kronecker product, which inherits positive semidefiniteness. The Hadamard exponential result then follows by observing that each term $R^{\circ k}/k!$ in the series expansion is PSD (by iterated application of the Schur product theorem), and a convergent sum of PSD matrices is PSD.

\inputleanmodule{OSforGFF.FrobeniusPositivity}
\inputleanmodule{OSforGFF.SchurProduct}
\inputleanmodule{OSforGFF.HadamardExp}
\inputleanmodule{OSforGFF.PositiveDefinite}
\inputleanmodule{OSforGFF.GaussianRBF}

\section{Nuclear Space}\label{sec:nuclear-space}

A small number of standard mathematical results are assumed as axioms (via \texttt{axiom}) because their proofs would require substantial Mathlib extensions. The nuclear space characterization is defined via Hilbert--Schmidt embedding, and the nuclearity of Schwartz space is assumed.

\inputleanmodule{OSforGFF.NuclearSpace}

% =============================================================================
% Chapter 2: Basic Definitions
% =============================================================================
\chapter{Basic Definitions}\label{ch:basic-definitions}

This chapter introduces the core type definitions and infrastructure for the Euclidean quantum field theory formalization: spacetime geometry, symmetry groups, test function spaces, Schwartz space integration, and generating functionals.

\section{Spacetime and Symmetries}\label{sec:spacetime-symmetries}

The spacetime is $\R^4$ with the standard Euclidean metric, modeled as $\texttt{EuclideanSpace}\ \R\ (\texttt{Fin}\ 4)$. The coordinate index 0 is the time direction. The Euclidean symmetry group $E(4) = O(4) \rtimes \R^4$ acts on spacetime by rotations and translations.

Key structures:
\begin{itemize}
\item \textbf{Spacetime:} $\R^4$ with the standard Euclidean inner product.
\item \textbf{Time coordinate:} The projection $x \mapsto x_0$.
\item \textbf{Euclidean group:} $E(4) = O(4) \rtimes \R^4$ acting on spacetime and test functions.
\item \textbf{Time reflection:} $\Theta: (t,\mathbf{x}) \mapsto (-t,\mathbf{x})$, a distinguished involution in $E(4)$.
\item \textbf{Spacetime decomposition:} The measurable equivalence $\R^4 \cong \R \times \R^3$ separating time and spatial coordinates.
\item \textbf{Positive time set:} The open half-space $\{x \in \R^4 : x_0 > 0\}$.
\end{itemize}

The Euclidean group is formalized as a structure containing an orthogonal matrix $R \in O(4)$ and a translation vector $t \in \R^4$. The group operations are:
\begin{itemize}
\item \emph{Multiplication:} $(R_1, t_1) \cdot (R_2, t_2) = (R_1 R_2,\, R_1 t_2 + t_1)$
\item \emph{Identity:} $(I, 0)$
\item \emph{Inverse:} $(R, t)^{-1} = (R^{-1},\, -R^{-1} t)$
\item \emph{Action:} $g \cdot x = Rx + t$
\end{itemize}
The formalization proves that this forms a group and that the action preserves Lebesgue measure.

\inputleanmodule{OSforGFF.Basic}
\inputleanmodule{OSforGFF.Euclidean}
\inputleanmodule{OSforGFF.DiscreteSymmetry}
\inputleanmodule{OSforGFF.SpacetimeDecomp}

\section{Test Function Spaces}\label{sec:test-function-spaces}

Test functions are Schwartz-class functions on $\R^4$, serving as the ``smearing functions'' that pair with distributional field configurations. The formalization uses Mathlib's \texttt{SchwartzMap} type. Key types:
\begin{itemize}
\item $\texttt{TestFunction} = \calS(\R^4, \R)$: Real Schwartz functions.
\item $\texttt{TestFunctionC} = \calS(\R^4, \C)$: Complex Schwartz functions.
\item $\texttt{PositiveTimeTestFunction}$: Real Schwartz functions supported in $\{x_0 > 0\}$.
\item $\texttt{FieldConfiguration} = \mathrm{WeakDual}\ \R\ \texttt{TestFunction}$: Distributional field configurations.
\end{itemize}

The complex test function space carries additional structure: complex conjugation, a star operation, and decomposition into real and imaginary parts.

Time translation is the one-parameter subgroup of the Euclidean group that shifts the time coordinate. It is formalized in detail because of its role in the ergodicity axiom (OS4). The key properties are:
\begin{itemize}
\item $T_{s+t} = T_s \circ T_t$ (semigroup law).
\item $T_0 = \mathrm{id}$ (identity).
\item $s \mapsto T_s f$ is continuous $\R \to \calS$ in the Schwartz topology.
\item The Lipschitz seminorm bound: $\|T_h f - f\|_{k,n} \le |h|(1+|h|)^k 2^k(\|f\|_{k,n+1}+\|f\|_{0,n+1}+1)$.
\end{itemize}
The Lipschitz seminorm bound is a fully proved result (no axioms) central to proving continuity of the time translation in the Schwartz topology. The proof uses the mean value theorem, Peetre's inequality for polynomial weight shifting, and careful norm estimates on iterated derivatives.

\inputleanmodule{OSforGFF.ComplexTestFunction}
\inputleanmodule{OSforGFF.PositiveTimeTestFunction_real}
\inputleanmodule{OSforGFF.TimeTranslation}

\section{Schwartz Space Integration}\label{sec:schwartz-integration}

Products of Schwartz functions with singular kernels arise throughout the construction. This section provides integrability results for such products, enabling the use of Fubini's theorem and change-of-variables.

The key technical result is the Tonelli-type identity for spacetime integrals: when $K(t_1, t_2)$ is a bounded measurable kernel depending only on time coordinates,
$$\iint \|f(x)\|\, \|g(y)\|\, K(x_0, y_0)\, dx\, dy = \iint K(t_1, t_2)\, G_f(t_1)\, G_g(t_2)\, dt_1\, dt_2$$
where $G_f(t) = \int_{\R^3} \|f(t, v)\|\, dv$ is the spatial marginal.

For positive-time-supported Schwartz functions, the vanishing at $t = 0$ gives the bound $|f(t,x)| \le Ct/(1+\|x\|)^4$ for $t > 0$, which combines the fundamental theorem of calculus with rapid spatial decay.

\inputleanmodule{OSforGFF.SchwartzProdIntegrable}
\inputleanmodule{OSforGFF.SchwartzTonelli}

\section{Generating Functionals}\label{sec:generating-functionals}

The generating functional $Z[J] = \int e^{i\langle\omega,J\rangle}\, d\mu(\omega)$ encodes all correlation functions of the field theory. For a Gaussian measure with covariance $C$, it takes the explicit form $Z[J] = e^{-\frac{1}{2}C(J,J)}$.

The Schwinger $n$-point functions are defined as
$$S_n(f_1, \ldots, f_n) = \int \langle\omega,f_1\rangle \cdots \langle\omega,f_n\rangle\, d\mu(\omega)$$
For the 2-point function, $S_2(f,g) = \int \langle\omega,f\rangle\, \langle\omega,g\rangle\, d\mu$.

The Gaussian property requires $Z[J] = e^{-\frac{1}{2}S_2(J,J)}$, i.e., all correlation functions are determined by the 2-point function via Wick's theorem.

The smeared-to-distributional 2-point function is recovered by expressing the smeared 2-point function as a double convolution of bump functions with the kernel $C$, then applying the double mollifier convergence theorem from Section~\ref{sec:functional-analysis}. As the bump support shrinks to zero, the smeared function converges pointwise to $C(x)$ for $x \ne 0$.

\inputleanmodule{OSforGFF.Schwinger}
\inputleanmodule{OSforGFF.SchwingerTwoPointFunction}

% =============================================================================
% Chapter 3: Free Covariance
% =============================================================================
\chapter{Free Covariance}\label{ch:free-covariance}

The free massive scalar field propagator $C(x, y)$ and its properties form the analytical core of the GFF construction. The construction proceeds from momentum space through Fourier transform to position space, establishing all the analytic properties needed for the OS axioms.

\section{Momentum Space and Schwinger Representation}\label{sec:covariance-momentum}

The free Euclidean propagator for a massive scalar field with mass $m > 0$ in $d = 4$ dimensions is defined in momentum space as
$$\hat{C}(k) = \frac{1}{\|k\|^2 + m^2}$$
The position-space kernel is obtained by inverse Fourier transform:
$$C(x - y) = \frac{1}{(2\pi)^4} \int_{\R^4} \frac{e^{ik\cdot(x-y)}}{\|k\|^2 + m^2}\, d^4k$$
In 4D, this evaluates to the Bessel function representation:
$$C(x, y) = \frac{m}{4\pi^2 \|x - y\|} K_1(m\|x - y\|)$$
where $K_1$ is the modified Bessel function of the second kind.

Key properties of the propagator:
\begin{itemize}
\item \emph{Positivity:} $C(x, y) > 0$ for $x \ne y$.
\item \emph{Symmetry:} $C(x, y) = C(y, x)$.
\item \emph{Euclidean invariance:} $C(gx, gy) = C(x, y)$ for all $g \in E(4)$.
\item \emph{Singularity:} $C(x, y) \sim \mathrm{const}/\|x-y\|^2$ as $x \to y$.
\item \emph{Exponential decay:} $C(x, y) \sim \mathrm{const} \cdot e^{-m\|x-y\|}/\|x-y\|^{3/2}$ as $\|x-y\| \to \infty$.
\end{itemize}

The propagator admits a proper-time (Schwinger) representation:
$$\frac{1}{\|k\|^2 + m^2} = \int_0^{\infty} e^{-t(\|k\|^2 + m^2)}\, dt$$
Performing the Fourier transform under the proper-time integral yields the heat kernel representation:
$$C(x, y) = \int_0^{\infty} e^{-tm^2} H_t(\|x-y\|)\, dt$$
where $H_t(r) = \frac{1}{16\pi^2 t^2} e^{-r^2/(4t)}$ is the 4D heat kernel.

To rigorously justify Fubini-type exchanges of integration, a Gaussian regulator $e^{-\alpha\|k\|^2}$ is introduced:
$$C_\alpha(x, y) = \int \frac{e^{-\alpha\|k\|^2} e^{ik\cdot(x-y)}}{\|k\|^2 + m^2} \frac{dk}{(2\pi)^4}$$
This makes all integrals absolutely convergent. The true covariance is recovered as $\alpha \to 0^+$.

The Schwinger representation is proved in five steps:
\begin{enumerate}
\item Evaluate $\int_0^\infty e^{-t(a^2+m^2)}\, dt = 1/(a^2+m^2)$ for $a^2 = \|k\|^2$.
\item Exchange the $k$-integral and $t$-integral (Fubini, justified by the regulator).
\item The inner $k$-integral is a Gaussian integral giving the heat kernel: $\int e^{-t\|k\|^2 - ik\cdot z}\, dk = (2\pi)^d H_t(\|z\|)$.
\item Complete the square to obtain the Schwinger representation of $C(x,y)$.
\item Evaluate via the Laplace integral to get the Bessel form.
\end{enumerate}

\inputleanmodule{OSforGFF.CovarianceMomentum}

\section{Parseval Identity and Positive Semidefiniteness}\label{sec:parseval}

For test functions $f, g \in \calS(\R^4, \C)$, the bilinear covariance form is:
$$\langle f, Cg\rangle = \iint \overline{f(x)}\, C(x-y)\, g(y)\, dx\, dy$$
The \emph{Parseval identity} relates this to momentum space:
$$\mathrm{Re}\langle f, Cf\rangle = \int \frac{|\hat{f}(k)|^2}{\|k\|^2 + m^2}\, dk$$
This immediately gives positive semi-definiteness: $\mathrm{Re}\langle f, Cf\rangle \ge 0$ since the integrand is non-negative.

The Parseval identity is proved in four stages:
\begin{enumerate}
\item \emph{Regulated Fubini factorization.} For the regulated integral $\langle f, C_\alpha f\rangle$, the Gaussian regulator $e^{-\alpha\|k\|^2}$ makes the triple integral $(x, y, k)$ absolutely convergent. Apply Fubini to reorder integration.
\item \emph{Change of variables.} Rescale from physics Fourier conventions ($\int e^{ikx}$) to Mathlib conventions ($\int e^{-2\pi i k \cdot x}$), absorbing factors of $(2\pi)^d$.
\item \emph{Monotone convergence.} As $\alpha \to 0^+$, the regulated integrand increases monotonically to the unregulated integrand. The dominated convergence theorem (with dominator $|\hat{f}|^2/m^2$) gives convergence.
\item \emph{Bilinear extension.} Extend from $\langle f, Cf\rangle$ to $\langle f, Cg\rangle$ via the polarization identity.
\end{enumerate}

\inputleanmodule{OSforGFF.Parseval}

\section{Covariance Properties and Hilbert Space Embedding}\label{sec:covariance-properties}

The covariance bilinear form has important structural properties: it is symmetric ($\langle f, Cg\rangle = \langle g, Cf\rangle$), additive and homogeneous in each argument, Euclidean-invariant ($\langle g\cdot f, C(g\cdot h)\rangle = \langle f, Ch\rangle$), and invariant under time reflection ($C(\Theta x, \Theta y) = C(x, y)$).

The covariance form factors through a Hilbert space. Define the map $T\colon \texttt{TestFunction} \to L^2$ by
$$Tf(k) = \hat{f}(k) \sqrt{\frac{1}{\|k\|^2 + m^2}}$$
Then $C(f, f) = \|Tf\|^2$, establishing the Hilbert space embedding required by the Minlos theorem.

For reflection positivity (Chapter~\ref{ch:os3}), the covariance restricted to real test functions has additional properties:
\begin{itemize}
\item $C(\Theta f, \Theta g) = C(f, g)$ (reflection invariance).
\item $C(\Theta f, g) = C(\Theta g, f)$ (reflection cross-symmetry).
\item Linearity of $\sum_i c_i\, C(\Theta f_i, g)$ in the coefficients $c_i$.
\end{itemize}

\inputleanmodule{OSforGFF.Covariance}
\inputleanmodule{OSforGFF.CovarianceR}

% =============================================================================
% Chapter 4: Gaussian Measure Construction
% =============================================================================
\chapter{Gaussian Measure Construction}\label{ch:gaussian-measure}

This chapter constructs the Gaussian Free Field (GFF) measure on the space of distributional field configurations and verifies its Gaussian property. The GFF measure $\mu_{\mathrm{GFF}}$ is the probability measure on the space of tempered distributions $\calS'(\R^4)$ characterized by its characteristic functional:
$$\int e^{i\langle\omega,f\rangle}\, d\mu(\omega) = e^{-\frac{1}{2}C(f,f)}$$
where $C(f, g)$ is the free covariance bilinear form.

\section{Minlos Theorem}\label{sec:minlos}

The construction uses the \emph{Minlos theorem}: a continuous, positive-definite functional on a nuclear space is the characteristic functional of a unique probability measure on the dual space.

The Minlos theorem requires three inputs:
\begin{enumerate}
\item \textbf{Positive definiteness of $e^{-\frac{1}{2}C(f,f)}$:} This follows from the Hilbert space embedding $C(f,f) = \|Tf\|^2$, combined with the abstract result that $e^{-\frac{1}{2}\|h\|^2}$ is positive-definite on any inner product space (proved via the Schur product theorem and Hadamard exponential).
\item \textbf{Nuclearity of Schwartz space:} Assumed as an axiom. This is a standard result whose proof requires the Hilbert--Schmidt theory of Schwartz seminorms.
\item \textbf{Continuity of $f \mapsto C(f,f)$:} Proved via the continuity of the embedding map $T$ and the fact that the $L^2$ norm is continuous.
\end{enumerate}

\inputleanmodule{OSforGFF.Minlos}

\section{Analytic Properties}\label{sec:analytic-properties}

The negation map $\omega \mapsto -\omega$ is measurable, and the GFF measure is invariant under it. This yields the vanishing of the first moment: $\int\langle\omega,a\rangle\, d\mu = 0$ for all test functions $a$.

\inputleanmodule{OSforGFF.MinlosAnalytic}

\section{GFF Measure Construction}\label{sec:gff-construction}

Apply the Minlos theorem to the free covariance to obtain a probability measure $\mu$ on $\texttt{FieldConfiguration}$ satisfying the characteristic function identity. The construction yields:
\begin{itemize}
\item \textbf{Centered:} $\int\langle\omega,f\rangle\, d\mu = 0$ for all $f$.
\item \textbf{Second moment:} $\int\langle\omega,\phi\rangle^2\, d\mu = C(\phi,\phi)$.
\item \textbf{Gaussian pairing:} The law of $\langle\omega,\phi\rangle$ is $\mathcal{N}(0, C(\phi,\phi))$.
\item \textbf{Moment finiteness:} $\langle\omega,\phi\rangle \in L^p$ for all $p < \infty$.
\item \textbf{Exponential integrability:} $e^{\alpha\langle\omega,\phi\rangle^2}$ is integrable for small $\alpha > 0$ (Fernique's theorem).
\end{itemize}

\inputleanmodule{OSforGFF.GFFMconstruct}

\section{Gaussian Moments}\label{sec:gaussian-moments}

The integrability of $\langle\omega,\phi\rangle\langle\omega,\psi\rangle$ for complex test functions follows from:
\begin{enumerate}
\item Fernique's theorem gives $e^{\alpha\langle\omega,\phi\rangle^2}$ integrable for small $\alpha > 0$.
\item From exp-square integrability, derive $L^p$ membership for all $p < \infty$ via H\"older's inequality.
\item Cauchy--Schwarz: $|\langle\omega,\phi\rangle\langle\omega,\psi\rangle| \le \tfrac{1}{2}(|\langle\omega,\phi\rangle|^2 + |\langle\omega,\psi\rangle|^2)$.
\end{enumerate}

\inputleanmodule{OSforGFF.GaussianMoments}

\section{Gaussianity Verification}\label{sec:gaussianity}

The extension from real to complex characteristic functions is the most technically involved step. For fixed real test functions $f, g$, both sides of the identity $Z(z_0, z_1) = e^{-\frac{1}{2}C(z_0 f + z_1 g,\, z_0 f + z_1 g)}$ are entire functions of $(z_0, z_1) \in \C^2$.
\begin{itemize}
\item \emph{Left side:} The integral $\int \exp(i\langle\omega, z_0 f + z_1 g\rangle)\, d\mu$ is analytic by Fernique's theorem (sub-Gaussian integrability).
\item \emph{Right side:} Analytic because it is the composition of the exponential with a polynomial.
\end{itemize}
They agree on $\R^2$ by the real characteristic functional identity. By analytic continuation (the identity theorem, applied slice by slice), they agree on $\C^2$. Any complex test function $J$ can be decomposed as $z_0 f + z_1 g$ where $f = \mathrm{Re}(J)$, $g = \mathrm{Im}(J)$, $z_0 = 1$, $z_1 = i$.

From the complex Gaussian identity $Z[J] = e^{-\frac{1}{2}S_2(J,J)}$, the 2-point function is extracted by differentiating twice: $S_2(f, g) = C(f, g)$.

\inputleanmodule{OSforGFF.GFFIsGaussian}
\inputleanmodule{OSforGFF.GaussianFreeField}

% =============================================================================
% Chapter 5: OS Axiom Definitions
% =============================================================================
\chapter{OS Axiom Definitions}\label{ch:os-axiom-definitions}

The Osterwalder--Schrader (OS) axioms characterize Euclidean quantum field theories that admit analytic continuation to relativistic (Minkowski) QFTs via the Wick rotation. They were introduced by Osterwalder and Schrader in 1973--1975, building on earlier work of Nelson and Symanzik.

The axioms are stated in terms of the \emph{generating functional} (or characteristic functional):
$$Z[f] = \int_{\Omega} e^{i\langle \omega, f \rangle}\, d\mu(\omega)$$
where $\Omega = \calS'(\R^4)$ is the space of tempered distributions (field configurations), $\mu$ is a probability measure on $\Omega$, $f$ is a test function in $\calS(\R^4)$, and $\langle \omega, f \rangle$ denotes the distributional pairing.

This project follows the Glimm--Jaffe formulation and proves the axioms for the Gaussian Free Field with mass $m > 0$ in 4-dimensional Euclidean spacetime.

\section{OS0: Analyticity}\label{sec:os0-def}

For any finite collection of complex test functions $J_1, \ldots, J_n \in \calS(\R^4, \C)$, the map
$$z = (z_1, \ldots, z_n) \mapsto Z\left[\sum_{i=1}^n z_i J_i\right]$$
is analytic on $\C^n$. Analyticity ensures that the Schwinger functions are well-defined as functional derivatives of $Z$, and that the theory can be analytically continued from Euclidean to Minkowski signature via the Wick rotation.

\section{OS1: Regularity}\label{sec:os1-def}

There exist $p \in [1, 2]$ and $c > 0$ such that for all $f \in \calS(\R^4, \C)$:
$$|Z[f]| \le \exp\left(c \cdot \left(\|f\|_{L^1} + \|f\|_{L^p}^p\right)\right)$$
When $p = 2$, the two-point Schwinger function $S_2(x)$ must be locally integrable. This bound ensures the field theory measure is well-defined and the Schwinger functions are tempered distributions.

\section{OS2: Euclidean Invariance}\label{sec:os2-def}

For all $g \in E(4)$ and $f \in \calS(\R^4, \C)$:
$$Z[g \cdot f] = Z[f]$$
where $(g \cdot f)(x) = f(g^{-1} x)$ is the pullback action. The vacuum state respects Euclidean symmetry. After Wick rotation, this becomes Poincar\'e invariance of the Minkowski theory.

\section{OS3: Reflection Positivity}\label{sec:os3-def}

Let $\Theta$ denote time reflection: $\Theta(t, \vec{x}) = (-t, \vec{x})$. For test functions $f_1, \ldots, f_n$ supported in the positive time half-space $\{t > 0\}$ and real coefficients $c_1, \ldots, c_n$:
$$\sum_{i,j} c_i\, c_j \cdot \mathrm{Re}\left(Z[f_i - \Theta f_j]\right) \ge 0$$
This is the key axiom for the OS reconstruction theorem. The positive semi-definite form becomes an inner product after quotienting by null vectors, yielding the physical Hilbert space of the QFT.

\section{OS4: Clustering and Ergodicity}\label{sec:os4-def}

The clustering axiom states that for test functions $f, g$ and any $\varepsilon > 0$, there exists $R > 0$ such that for spatial separation $\|a\| > R$:
$$|Z[f + T_a g] - Z[f] \cdot Z[g]| < \varepsilon$$
Distant regions of spacetime become statistically independent, which is equivalent to uniqueness of the vacuum state.

The ergodicity formulation states that for generating-function observables $A(\varphi) = \sum_j z_j e^{\langle\varphi, f_j\rangle}$, the time average converges to the expectation in $L^2(\mu)$:
$$\lim_{T \to \infty} \int_\Omega \left|\frac{1}{T}\int_0^T A(T_s \varphi)\, ds - \mathbb{E}_\mu[A]\right|^2 d\mu(\omega) = 0$$

A quantitative polynomial clustering variant is also established: for any $\alpha > 0$, there exists $c \ge 0$ such that for all $s \ge 0$:
$$\left|\mathbb{E}_\mu\left[e^{\langle\varphi,f\rangle + \langle T_s\varphi, g\rangle}\right] - \mathbb{E}_\mu\left[e^{\langle\varphi,f\rangle}\right]\mathbb{E}_\mu\left[e^{\langle\varphi,g\rangle}\right]\right| \le c\, (1 + s)^{-\alpha}$$

\inputleanmodule{OSforGFF.OS_Axioms}

% =============================================================================
% Chapter 6: OS0 — Analyticity
% =============================================================================
\chapter{OS0 --- Analyticity}\label{ch:os0}

OS0 (Analyticity) is the first Osterwalder--Schrader axiom. It states that the generating functional $Z[f]$ depends analytically on the test function arguments. This is essential for the Wick rotation connecting Euclidean and Minkowski signatures, for defining correlation functions as functional derivatives of $Z$, and for perturbation theory via Taylor expansions.

\section{Proof Strategy}\label{sec:os0-strategy}

The proof applies a \emph{holomorphic integral theorem}: if the integrand $f(z, \omega) = \exp(i\sum_i z_i \langle\omega, J_i\rangle)$ satisfies appropriate measurability, analyticity, integrability, and derivative bound conditions in $z$ and $\omega$, then the integral $F(z) = \int f(z, \omega)\, d\mu(\omega)$ is holomorphic. Goursat's theorem in $n$ dimensions then converts holomorphy to analyticity.

The five hypotheses verified are:
\begin{enumerate}
\item \emph{Measurability} of $\omega \mapsto f(z, \omega)$ for each $z$: The evaluation functional $\omega \mapsto \langle\omega, f\rangle$ is continuous on the weak-$*$ dual $\calS'$, so compositions with measurable operations preserve measurability.

\item \emph{Analyticity} of $z \mapsto f(z, \omega)$ for each $\omega$: The sum $\sum_i z_i \langle\omega, J_i\rangle$ is linear in $z$ (hence analytic), multiplication by $i$ and the exponential are entire, and compositions of analytic functions are analytic.

\item \emph{Integrability} of $f(z, \cdot)$ for each $z$: For complex test functions $f = f_{\mathrm{re}} + i f_{\mathrm{im}}$, we have $\|\exp(i\langle\omega, f\rangle)\| = \exp(-\omega(f_{\mathrm{im}}))$. Integrability requires Fernique's theorem: for Gaussian measures, $\exp(\alpha \langle\omega, g\rangle^2)$ is integrable for sufficiently small $\alpha > 0$.

\item \emph{Measurability of the Fr\'echet derivative} in $\omega$: The derivative has explicit form involving continuous operations, hence is measurable.

\item \emph{Local integrable bound on the derivative}: An integrable dominating function is constructed using H\"older's inequality ($L^2 \times L^2 \to L^1$), Gaussian exponential integrability, and Gaussian polynomial moments.
\end{enumerate}

With all five hypotheses verified, the holomorphic integral theorem establishes analyticity. The proof requires one project-specific axiom: Goursat's theorem in $n$ dimensions (holomorphic implies analytic in $\C^n$), which is standard in several complex variables but not yet available in Mathlib.

\inputleanmodule{OSforGFF.OS0_GFF}

% =============================================================================
% Chapter 7: OS1 — Regularity
% =============================================================================
\chapter{OS1 --- Regularity}\label{ch:os1}

OS1 (Regularity) establishes that the generating functional $Z[f]$ has controlled growth as a function of the test function $f$. For the GFF with mass $m > 0$, the proof establishes the bound with $p = 2$ and $c = 1/(2m^2)$.

\section{Exponential Bound}\label{sec:os1-exponential}

The GFF generating functional has the closed form $Z[f] = \exp(-\tfrac{1}{2}\langle f, Cf\rangle_{\C})$. The proof proceeds in four steps.

\paragraph{Step 1: Norm via complex exponential.} Taking norms:
$$|Z[f]| = \exp\left(-\tfrac{1}{2}\mathrm{Re}\langle f, Cf\rangle_{\C}\right)$$
using $|e^z| = e^{\mathrm{Re}(z)}$ and the Gaussian structure of the GFF.

\paragraph{Step 2: Bound by imaginary part.} Decomposing $f = f_{\mathrm{re}} + if_{\mathrm{im}}$ and expanding the sesquilinear form:
$$\mathrm{Re}\langle f, Cf\rangle = \mathrm{Re}\langle f_{\mathrm{re}}, Cf_{\mathrm{re}}\rangle - \mathrm{Re}\langle f_{\mathrm{im}}, Cf_{\mathrm{im}}\rangle$$
The cross terms vanish because the covariance is real-valued and symmetric on real test functions. Since the covariance is positive semi-definite, $\mathrm{Re}\langle f_{\mathrm{re}}, Cf_{\mathrm{re}}\rangle \ge 0$, giving:
$$-\mathrm{Re}\langle f, Cf\rangle \le \mathrm{Re}\langle f_{\mathrm{im}}, Cf_{\mathrm{im}}\rangle$$

\paragraph{Step 3: Momentum space bound.} In momentum space, the covariance becomes multiplicative:
$$\langle g, Cg\rangle = \int \frac{|\hat{g}(k)|^2}{(2\pi)^2 \|k\|^2 + m^2}\, dk$$
The free propagator satisfies $1/((2\pi)^2\|k\|^2 + m^2) \le 1/m^2$. By the Plancherel theorem:
$$\langle g, Cg\rangle \le \frac{1}{m^2}\|g\|_{L^2}^2$$
Since $\|f_{\mathrm{im}}\|_{L^2} \le \|f\|_{L^2}$, combining gives:
$$|Z[f]| \le \exp\left(\frac{1}{2m^2}\|f\|_{L^2}^2\right)$$

\paragraph{Step 4: Add $L^1$ term.} Since $\|f\|_{L^1} \ge 0$, the bound is strengthened to match the OS1 form:
$$|Z[f]| \le \exp\left(\frac{1}{2m^2}\left(\|f\|_{L^1} + \|f\|_{L^2}^2\right)\right)$$

\section{Two-Point Local Integrability}\label{sec:os1-local-integ}

The GFF two-point function satisfies $|S_2(x)| \le C/\|x\|^2$ from Bessel function asymptotics ($K_1(z) \sim 1/z$ near the origin, exponential decay at infinity). Since $\|x\|^{-2}$ is locally integrable in 4 dimensions ($2 < 4$):
$$\int_{\|x\| < R} \|x\|^{-2}\, dx = |S^3| \cdot \int_0^R r^{-2} \cdot r^3\, dr = 2\pi^2 \cdot R^2/2 < \infty$$
Combined with the decay bound and measurability, this establishes the two-point integrability requirement.

The OS1 proof uses zero project-specific axioms. The decay bound and measurability are proved theorems derived from Bessel function analysis.

\inputleanmodule{OSforGFF.OS1_GFF}

% =============================================================================
% Chapter 8: OS2 — Euclidean Invariance
% =============================================================================
\chapter{OS2 --- Euclidean Invariance}\label{ch:os2}

OS2 (Euclidean Invariance) requires that the quantum field theory has no preferred direction or origin in Euclidean spacetime. After Wick rotation, Euclidean invariance becomes Poincar\'e invariance of the Minkowski theory.

\section{Proof Strategy}\label{sec:os2-strategy}

The proof factors into two parts:
\begin{enumerate}
\item \emph{General Gaussian result:} For any Gaussian measure whose covariance is Euclidean-invariant, $Z[g \cdot f] = Z[f]$.
\item \emph{GFF-specific:} The free GFF covariance depends only on $\|x - y\|$, which is preserved by Euclidean transformations.
\end{enumerate}

The free covariance kernel is $C(x, y) = \frac{m}{4\pi^2 \|x - y\|} K_1(m\|x - y\|)$. This depends only on $\|x - y\|$. Since Euclidean transformations preserve distances ($\|g \cdot x - g \cdot y\| = \|R(x - y)\| = \|x - y\|$), we have $C(g \cdot x, g \cdot y) = C(x, y)$.

Lebesgue measure on $\R^4$ is invariant under Euclidean transformations (translation invariance plus $|\det R| = 1$ for $R \in O(4)$). The proof constructs a measurable equivalence for each group element and proves it is measure-preserving, enabling the change of variables in integration.

The bilinear form invariance $\langle g \cdot f, C(g \cdot h)\rangle = \langle f, Ch\rangle$ proceeds by substituting the pullback, rewriting the kernel using Euclidean invariance, recognizing the integrand as a composition, and applying the measure-preserving change of variables.

The general theorem then derives OS2 from Gaussianity and covariance invariance:
$$Z[g \cdot f] = \exp\left(-\tfrac{1}{2}\langle g \cdot f, C(g \cdot f)\rangle\right) = \exp\left(-\tfrac{1}{2}\langle f, Cf\rangle\right) = Z[f]$$

This is the cleanest OS axiom proof, requiring zero project-specific axioms.

\inputleanmodule{OSforGFF.OS2_GFF}

% =============================================================================
% Chapter 9: OS3 — Reflection Positivity
% =============================================================================
\chapter{OS3 --- Reflection Positivity}\label{ch:os3}

Reflection positivity (OS3) is the most technically demanding of the Osterwalder--Schrader axioms. It is the Euclidean manifestation of \emph{unitarity} --- the requirement that quantum mechanics has positive probabilities and the Hilbert space has a positive-definite inner product. Without OS3, a Euclidean field theory cannot be analytically continued to define a physical relativistic QFT.

The proof has two main parts:
\begin{enumerate}
\item \textbf{Covariance reflection positivity via Fourier analysis:} Show $\mathrm{Re}\langle \Theta f, Cf \rangle \ge 0$ for each positive-time $f$ by rewriting the bilinear form in momentum space and factorizing into a squared norm.
\item \textbf{Matrix assembly via the Hadamard (Schur) product theorem:} Extend from a single test function to the general $n$-function case using Gaussian factorization and the fact that entrywise exponentials of PSD matrices remain PSD.
\end{enumerate}

\section{Covariance Reflection Positivity}\label{sec:os3-covariance-rp}

The goal is to show $\mathrm{Re}\langle\Theta f, f\rangle_C \ge 0$ for any complex test function $f$ with $f(x) = 0$ when $x_0 \le 0$.

\paragraph{Step 1: Schwinger representation.} Replace $C(\Theta x, y)$ by its proper-time (Schwinger) representation:
$$C(\Theta x, y) = \int_0^\infty e^{-sm^2} H(s, \|\Theta x - y\|)\, ds$$
where $H(s, r) = (4\pi s)^{-d/2} e^{-r^2/(4s)}$ is the heat kernel.

\paragraph{Step 2: Fourier decomposition.} Write the heat kernel as a Fourier integral and decompose $k = (k_0, \vec{k})$ to separate the time and spatial contributions. Three Fubini interchanges are proved with explicit integrability bounds.

\paragraph{Step 3: Evaluate the proper-time integral.} After separating momentum components, the $s$-integral evaluates analytically:
$$\int_0^\infty \sqrt{\pi/s}\, e^{-t^2/(4s)}\, e^{-s\omega^2}\, ds = \frac{\pi}{\omega}\, e^{-\omega|t|}$$
where $\omega = \sqrt{\|\vec{k}\|^2 + m^2}$. This yields the \emph{mixed representation}:
$$\langle\Theta f, f\rangle_C = \frac{1}{2(2\pi)^3} \int_{\R^3} \frac{1}{\omega}\, (\text{inner integral})\, d\vec{k}$$

\paragraph{Step 4: Factorization using positive-time support.} For positive-time test functions ($x_0 > 0$, $y_0 > 0$), we have $|-x_0 - y_0| = x_0 + y_0$, giving the key factorization $e^{-\omega(x_0 + y_0)} = e^{-\omega x_0} \cdot e^{-\omega y_0}$. The double $(x,y)$-integral then factors into $|F_\omega(\vec{k})|^2$ where $F_\omega$ is the \emph{weighted Laplace--Fourier transform}:
$$F_\omega(\vec{k}) = \int_{\R^4} f(x)\, e^{-\omega x_0}\, e^{-i\vec{k}\cdot\vec{x}}\, dx$$

\paragraph{Step 5: Non-negativity.} The final form is:
$$\langle\Theta f, f\rangle_C = \frac{1}{2(2\pi)^3} \int_{\R^3} \frac{1}{\omega(\vec{k})}\, |F_\omega(-\vec{k})|^2\, d\vec{k}$$
Since $\omega > 0$ (because $m > 0$) and $|F_\omega|^2 \ge 0$, the integrand is pointwise non-negative.

\inputleanmodule{OSforGFF.OS3_MixedRepInfra}
\inputleanmodule{OSforGFF.OS3_MixedRep}
\inputleanmodule{OSforGFF.OS3_CovarianceRP}

\section{Matrix Assembly}\label{sec:os3-matrix-assembly}

\paragraph{Step 6: Reflection covariance matrix is PSD.} Define $R_{ij} = \langle\Theta f_i, C f_j\rangle$. For any real vector $c$:
$$\sum_{ij} c_i c_j R_{ij} = \langle\Theta g, Cg\rangle \ge 0 \quad\text{where } g = \sum_i c_i f_i$$
using bilinearity and the single-function result.

\paragraph{Step 7: Hadamard exponential preserves PSD.} By the Schur product theorem, the entrywise exponential $E_{ij} = e^{R_{ij}}$ is PSD. The proof uses the Hadamard series $e^{R_{ij}} = \sum_{k=0}^\infty (R_{ij})^k/k!$, where each term is PSD by iterated Hadamard products, and the convergent sum of PSD matrices is PSD.

\paragraph{Step 8: Entry factorization.} The quadratic expansion gives:
$$C(f - \Theta g, f - \Theta g) = C(f,f) + C(g,g) - 2\langle\Theta f, Cg\rangle$$
yielding the entry factorization:
$$\mathrm{Re}\left(Z[f_i - \Theta f_j]\right) = Z[f_i] \cdot Z[f_j] \cdot e^{R_{ij}}$$
where $Z[f] = \exp(-C(f,f)/2) > 0$. Setting $y_i = c_i \cdot Z[f_i]$ and assembling:
$$\sum_{ij} c_i c_j\, \mathrm{Re}\, Z[f_i - \Theta f_j] = \sum_{ij} y_i y_j\, E_{ij} = y^T E y \ge 0$$

\inputleanmodule{OSforGFF.OS3_GFF}

% =============================================================================
% Chapter 10: OS4 — Clustering and Ergodicity
% =============================================================================
\chapter{OS4 --- Clustering and Ergodicity}\label{ch:os4}

The OS4 axiom comes in two related forms: \emph{clustering} (distant regions become statistically independent) and \emph{ergodicity} (time averages converge to ensemble averages). For the Gaussian Free Field, clustering follows from the decay of the free propagator, and ergodicity follows from clustering via variance bounds on time averages.

\section{MGF Infrastructure}\label{sec:os4-mgf}

The moment generating functional $M[f] = \mathbb{E}[e^{\langle\phi, f\rangle}]$ is related to the characteristic functional by analytic continuation. For the GFF, $M[f] = e^{\frac{1}{2}C(f,f)}$.

Key structural results:
\begin{itemize}
\item Time translation invariance: $C(T_sf, T_sg) = C(f,g)$, hence $Z[T_sf] = Z[f]$.
\item Joint MGF factorization: $\mathbb{E}[e^{\langle\phi,f+g\rangle}] = M[f] \cdot M[g] \cdot e^{C(f,g)}$.
\item Exponential estimate: $|e^x - 1| \le |x| e^{|x|}$.
\end{itemize}

\inputleanmodule{OSforGFF.OS4_MGF}

\section{Clustering}\label{sec:os4-clustering}

The clustering proof exploits the Gaussian factorization formula:
\begin{enumerate}
\item For the GFF, $Z[f] = \exp(-\tfrac{1}{2}C(f,f))$ for real test functions, so the bilinear expansion gives:
$$Z[f + T_a g] = Z[f] \cdot Z[T_a g] \cdot \exp(-S_2(f, T_a g))$$
where $S_2(f, T_a g) = C(f, T_a g)$ is the cross covariance.

\item By Euclidean invariance (OS2): $Z[T_a g] = Z[g]$.

\item The cross term decays as $\|a\| \to \infty$ because the propagator $C(x-y) \sim 1/\|x-y\|^2$ at large distances, and the Schwartz test functions provide rapid localization.

\item Since $|Z[f]| \le 1$ for real test functions and $|e^{-z} - 1| \le 2|z|$ for $|z| \le 1$, we conclude $|Z[f + T_ag] - Z[f]Z[g]| \le 2|S_2(f, T_ag)| \to 0$.
\end{enumerate}

The polynomial clustering variant with decay exponent $\alpha = 6$ uses the exponential decay of the massive propagator $C(z) \sim e^{-m\|z\|}$ combined with Schwartz decay. The bilinear translation decay theorem gives a polynomial bound $c(1+\|a\|)^{-\alpha}$ for any $\alpha > 0$.

\inputleanmodule{OSforGFF.OS4_Clustering}

\section{Ergodicity}\label{sec:os4-ergodicity}

The ergodicity proof proceeds through intermediate formulations:
$$\text{OS4'' (Polynomial Clustering, } \alpha=6\text{)} \implies \text{OS4' (Single-function Ergodicity)} \implies \text{OS4 (Full Ergodicity)}$$

The key steps from clustering to ergodicity are:
\begin{enumerate}
\item \textbf{Variance bound:} The variance of the time average is bounded by the double time integral of the covariance:
$$\mathrm{Var}\left(\frac{1}{T}\int_0^T A(T_s\phi)\, ds\right) \le \frac{1}{T^2}\int_0^T\int_0^T |\mathrm{Cov}(s,u)|\, ds\, du$$

\item \textbf{Polynomial decay of covariance:} From clustering, $|\mathrm{Cov}(s,u)| \le c(1+|s-u|)^{-\alpha}$.

\item \textbf{Double integral bound:} For $\alpha > 1$, the double integral satisfies $\int_0^T\int_0^T (1+|s-u|)^{-\alpha}\, ds\, du \le C \cdot T$.

\item \textbf{Variance vanishes:} Combining, the variance is $O(1/T) \to 0$ as $T \to \infty$.

\item \textbf{Extension to linear combinations:} The single generating function result extends to finite linear combinations $\sum_j z_j e^{\langle\phi, f_j\rangle}$ via the Cauchy--Schwarz inequality for weighted sums.
\end{enumerate}

Key technical results proved along the way:
\begin{itemize}
\item $L^2$ membership of time-translated exponentials: Uses the Gaussian MGF formula to show $\mathbb{E}[|e^{\langle T_s\phi, f\rangle}|^2] < \infty$.
\item Stationarity of product expectations: $\mathrm{Cov}(s,u)$ depends only on $s - u$, using time-translation invariance of the GFF covariance form.
\item Continuity of the covariance function: Uses stationarity, MGF factorization, and continuity of the covariance form under time translation.
\item Time average is in $L^2$: Uses the uniform $L^2$ bound (constant in $s$) and Fubini's theorem.
\end{itemize}

\inputleanmodule{OSforGFF.OS4_Ergodicity}

% =============================================================================
% Chapter 11: Master Theorem
% =============================================================================
\chapter{Master Theorem}\label{ch:master-theorem}

The master theorem assembles the individual axiom proofs into a single statement: the Gaussian Free Field with mass $m > 0$ in 4-dimensional Euclidean spacetime satisfies all Osterwalder--Schrader axioms.

For any mass parameter $m > 0$, the GFF probability measure $\mu_{\mathrm{GFF}}(m)$ satisfies:
\begin{itemize}
\item \textbf{OS0} (Analyticity): The generating functional is analytic in the test functions (Chapter~\ref{ch:os0}).
\item \textbf{OS1} (Regularity): The generating functional satisfies exponential bounds (Chapter~\ref{ch:os1}).
\item \textbf{OS2} (Euclidean Invariance): The measure is invariant under 4D Euclidean transformations (Chapter~\ref{ch:os2}).
\item \textbf{OS3} (Reflection Positivity): The quadratic form $\sum_{ij} c_i c_j\, \mathrm{Re}(Z[f_i - \Theta f_j]) \ge 0$ (Chapter~\ref{ch:os3}).
\item \textbf{OS4} (Clustering): Distant test functions become statistically independent (Chapter~\ref{ch:os4}).
\item \textbf{OS4} (Ergodicity): Time averages of generating-function observables converge to ensemble averages (Chapter~\ref{ch:os4}).
\end{itemize}

The master theorem in \texttt{GFFmaster.lean} is a short assembly file that imports and combines six independently proved results. Each axiom feeds in via a dedicated proof file, as detailed in the preceding chapters.

The complete proof depends on three axioms declared with the \texttt{axiom} keyword in Lean:
\begin{enumerate}
\item \textbf{Minlos theorem:} A continuous positive-definite normalized functional on a nuclear space is the characteristic functional of a unique probability measure.
\item \textbf{Goursat's theorem in $n$ dimensions:} A holomorphic function $f : \C^n \supset U \to \C$ is analytic (representable by a convergent power series).
\item \textbf{Nuclearity of Schwartz space:} $\calS(\R^n, F)$ is a nuclear topological vector space.
\end{enumerate}
All three are well-established theorems in functional analysis whose full Lean formalization was deferred.

\inputleanmodule{OSforGFF.GFFmaster}

% =============================================================================
% Appendix: Dimension Dependence
% =============================================================================
\appendix
\chapter{Dimension Dependence}\label{ch:dimension-dependence}

This appendix inventories where the spacetime dimension $d = 4$ (and spatial dimension $d-1 = 3$) enters the proofs. The project defines \texttt{STDimension := 4} in \texttt{Basic.lean}; changing this value would require updates to every item listed below as \emph{essential} or \emph{spatial}.

\section{Essential ($d=4$) Dependencies}\label{sec:dim-essential}

These formulas evaluate differently in other dimensions.

\begin{description}
\item[Heat kernel normalization:] The heat kernel uses $(4\pi t)^{-d/2}$, which evaluates to $1/(16\pi^2 t^2)$ for $d=4$.
\item[Bessel representation:] The closed-form covariance $C(x,y) = \frac{m}{4\pi^2 r} K_1(mr)$ is specific to $d=4$. In general dimension $d$, the covariance involves $K_{d/2-1}$.
\item[Schwinger evaluation:] The connection between the proper-time integral and the Bessel form uses the $d=4$ heat kernel.
\item[Plancherel scaling:] The factor $(2\pi)^d$ in Fourier convention changes.
\item[Mixed representation normalization:] The identity $(2\pi)^4 / (2\pi) = (2\pi)^3$ for factoring the mixed representation.
\end{description}

\section{Spatial ($d-1=3$) Dependencies}\label{sec:dim-spatial}

These use the spatial dimension $d-1 = 3$ for integrability.

\begin{description}
\item[Polynomial decay integrability:] $(1+\|x\|)^{-4}$ is integrable in $\R^3$ because the decay rate $4 > d-1 = 3$. In general dimension, the required decay rate would be $> d-1$.
\item[Spatial marginal bounds:] The Schwartz product integrability section uses spatial marginals in $\R^3$.
\item[Local integrability:] The condition $\alpha = 2 < d = 4$ in the two-point local integrability. In general dimension $d$, the singularity is $\|x\|^{-(d-2)}$, so the condition becomes $d-2 < d$, which always holds.
\item[Norm decomposition:] The expansion $\|k\|^2 = k_0^2 + k_1^2 + k_2^2 + k_3^2$ is coupled to $d=4$.
\end{description}

\section{Structural (Dimension-Agnostic) Components}\label{sec:dim-structural}

The following components reference \texttt{STDimension} or \texttt{SpaceTime} but work for any $d \ge 2$: the OS axiom definitions, the Euclidean group structure, time reflection, time translation, spacetime decomposition, complex test function operations, positive-time support, Schwinger functions, the Minlos theorem, the Gaussian measure construction, moment computations, all five OS axiom proofs (OS0--OS4), $L^2$ time integral estimates, the master theorem assembly, 1D Fourier transform identities, Laplace integrals, Hadamard exponentials, Schur product theorem, positive-definite function theory, Gaussian RBF, polynomial decay estimates, bilinear translation decay, and Tonelli for spacetime integrals.

\section{Generalization Notes}\label{sec:dim-generalization}

To port the project to a different dimension $d$:
\begin{enumerate}
\item Change \texttt{STDimension} in \texttt{Basic.lean}.
\item Replace the Bessel $K_1$ representation with the general $K_{d/2-1}$ form.
\item Update the heat kernel normalization $(4\pi t)^{-d/2}$.
\item Update Plancherel scaling factors $(2\pi)^d$.
\item Verify spatial integrability: decay rate must exceed $d-1$.
\item Replace the norm decomposition with the appropriate variant.
\end{enumerate}

\end{document}
